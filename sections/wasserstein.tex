\section{Constructing the queuing model}\label{sec:wasserstein}

The simplest model using the data available is an \(M|M|c\) queue with multiple
classes. In this model, the following assumptions are made:

\begin{enumerate}
    \item Inter-arrival and service times of patients are each exponential with
        some mean.
    \item There are $c$ servers available to arriving patients at a single node
        representing the overall resource availability at the hospital.
    \item There is no queue or system capacity.
    \item A first-in first-out service policy is implemented.
\end{enumerate}

Each group of patients has its own arrival distribution. The parameter of this
distribution is taken to be the reciprocal of the mean inter-arrival times for
that group.

Like arrivals, each group of patients has its own service time distribution.
This will be calculated approximately via the length of a patient's stay.
The length of stay is the total time spent in the system. Without full details
of the process order or idle periods during a spell, some assumption must be
made about the true `service' time in relation to the time spent in hospital.
This work considers the mean service time, \(\frac{1}{\mu}\), to be
proportional to the mean total system time, \(\frac{1}{\phi}\), such that:

\begin{equation}
    \mu = p \phi
\end{equation}

where \(p \in \interval[open left]{0}{1}\) is some parameter to be determined
for each group.

As the full details of how the patients move through the hospital system, and
the details of the system itself, are unknown, an appropriate number of servers
\(c\) must be found as well as \(p\).

In order to evaluate appropriate values of each \(p\) and the value of \(c\),
the system is simulated across of parameter space. Then, for each set of
parameters, the total time distribution is compared with that in the available
data via the (first) Wasserstein distance. This distance measures the
approximate `minimal work' required to move between two probability
distributions where `work' can be loosely defined as the product of how much of
the distribution's mass must be to be moved with the distance it must be moved
by. More formally, the Wasserstein distance between two probability
distributions \(U\) and \(V\) is defined as:

\begin{equation}\label{eq:wasserstein}
    W(U, V) = \int_{0}^{1} \left\vert F^{-1}(t) - G^{-1}(t) \right\vert dt
\end{equation}

where \(F\) and \(G\) are the cumulative density functions of \(U\) and \(V\)
respectively. A proof of~\eqref{eq:wasserstein} is presented in~\cite{RTC17}.

Then the parameter set with the smallest mean distance over a number of runs is
taken to be the most appropriate.

Reiterate the objective of the paper --- to model a COPD ward within a hospital
--- and draw attention to lack of fine-grain data. Lead into how this can be
overcome with the Wasserstein distance (a lot of this has been written up in
\texttt{nbs/wasserstein.ipynb}). A brief summary of how the parameter set is
chosen and a nice image of the queue we are building. Close out the section with
best and worst case parameter set plots.
