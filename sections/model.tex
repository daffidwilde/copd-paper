\section{Constructing the queuing model}\label{sec:model}

% TODO What exactly is the problem and how can it be addressed?

Owing to a lack of available data on the system and its patients, the options
for the queuing model used are limited compared to those employed in some modern
works. However, there is a precedent for simplifying healthcare systems to a
single node with, for example, parallel servers that emulate resource
availability.~\cite{Steins2013}~and~\cite{Williams2015} provide good examples
of how this approach, when paired with discrete event simulation, can expose the
resource needs of a system beyond deterministic queuing theory models. In
particular,~\citeauthor{Williams2015} show how a single node, multiple server
queue can be used to accurately predict bed capacity and length of stay
distributions in a critical care unit using administrative data.

To follow in the suit of recent literature, a single node using a \(M|M|c\)
queue is employed to model a hypothetical ward of patients presenting COPD.\ In
addition to this, the grouping found in Section~\ref{subsec:overview} provides
a set of patient classes in the queue. Under this model, the following
assumptions are made:
\begin{enumerate}
    \item Inter-arrival and service times of patients are each exponentially
        distributed with some mean. This is in spite of the system time
        distributions shown in Figure~\ref{fig:clusters_los} in order to
        simplify the model parameterisation.
    \item There are \(c\) servers available to arriving patients at the node
        representing the overall resource availability including bed capacity
        and medical staff.
    \item There is no queue or system capacity. In~\cite{Williams2015}, a
        queue capacity of zero is set under the assumption that any surplus
        arrivals would be sent to another suitable ward or unit. As this
        hypothetical ward represents COPD patients potentially throughout a
        hospital, this assumption is not held.
    \item Without the availability of expert clinical knowledge, a first-in
        first-out service policy is employed in lieu of some patient priority 
        framework.
\end{enumerate}

Each group of patients has its own arrival distribution. The parameter of this
distribution is taken to be the reciprocal of the mean inter-arrival times for
that group.

Like arrivals, each group of patients has its own service time distribution.
Without full details of the process order or idle periods during a spell, some
assumption must be made about the true `service' time of a patient in hospital.
It is assumed here that the mean service time of a group of patients may be
approximated via their mean length of stay, i.e.\ the mean time spent in the
system. For simplicity, this work considers the mean service time,
\(\frac{1}{\mu}\), to be directly proportional to the mean total system time,
\(\frac{1}{\phi}\), such that:
\begin{equation}
    \mu = p \phi
\end{equation}

\noindent where \(p \in \interval[open left]{0}{1}\) is some parameter to be
determined for each group, denoted by \(p_i\) for group \(i\).

In order to evaluate appropriate values of each \(p_i\) and the value of \(c\),
the system is simulated across a parameter space. The objective then is to
determine which set of parameters provides the most realistic analogue of the
observed data. Given that the length of stay is one of the few ground truths
available in the provided data, and given that length of stay and resource
availability are connected, the output of this model is the simulated length of
stay. With this simulated data, a parameter set can be evaluated by comparing
its associated system time distribution with the real length of stay
distribution.

The statistical comparison of two or more distributions can be done in a number
of ways. Such methods include the Kolmogorov-Smirnov test, a variety of
discrepancy approaches such as summed mean-squared error, and \(f\)-divergences.
A popular choice amongst the latter group (which may be considered
distance-like) is the Kullback-Leibler divergence which measures relative
information entropy from one probability distribution to
another~\cite{Kullback1951}. The key issue with many of these methods is that
they lack interpretability which is paramount when conveying information to
stakeholders. Interpretability not just from explaining how something works but
how its results may be explained also.

As such, a reasonable candidate is the (first) Wasserstein metric, also known as
the `earth mover' or `digger' distance~\cite{Vaserstein1969}. The Wasserstein
metric satisfies the conditions of a formal mathematical metric (like the
typical Euclidean distance), and its values take the units of the distributions
under comparison (in this case: days). Both of these characteristics can aid
understanding and explanation. In simple terms, the distance measures the
approximate `minimal work' required to move between two probability
distributions where `work' can be loosely defined as the product of how much of
the distribution's mass is to be moved and the distance it must be moved
by. More formally, the Wasserstein distance between two probability
distributions \(U\) and \(V\) is defined as:
\begin{equation}\label{eq:wasserstein}
    W(U, V) = \int_{0}^{1} \left\vert F^{-1}(t) - G^{-1}(t) \right\vert dt
\end{equation}

\noindent where \(F\) and \(G\) are the cumulative density functions of \(U\)
and \(V\) respectively. A proof of~\eqref{eq:wasserstein} is presented
in~\cite{Ramdas2017}. The parameter set with the smallest maximum distance
between any cluster's simulated system time distribution and the overall
observed length of stay distribution is then taken to be the most appropriate.

Specifically, the parameter sweep included values of each \(p_i\) from \(0.5\)
to \(1\) with a granularity of \(10^{-1}\) and values of \(c\) from \(40\) to
\(60\) at steps of \(5\).  % TODO Update these values when the new results come back
These choices were informed by the assumptions of the
model and formative analysis to reduce the parameter space given the
computational resources required to conduct the simulations. Each parameter set
was repeated \(50\) times with each simulation running for four years of virtual
time. The warm-up and cool-down periods were taken to be approximately one year
each leaving two years of simulated data from each repetition.

\begin{figure}
    \centering%
    \includegraphics[width=\imgwidth]{best_params}
    \caption{A histogram of the simulated and observed length of stay data for
             the best parameter set.}\label{fig:best_params}
\end{figure}

\begin{figure}
    \centering%
    \includegraphics[width=\imgwidth]{worst_params}
    \caption{A histogram of the simulated and observed length of stay data for
             the worst parameter set.}\label{fig:worst_params}
\end{figure}

The results of this parameter sweep can be summarised in
Figures~\ref{fig:best_params}~and~\ref{fig:worst_params}. Each figure shows a
comparison of the observed lengths of stay across all groups and the newly
simulated data with the best and worst parameter sets respectively. It can be
seen that, in the best case, a very close fit has been found. Meanwhile,
Figure~\ref{fig:worst_params} highlights the importance of good parameter
estimation under this model since the likelihood of short-stay patient arrivals
has been inflated disproportionately against the tail of the distribution.
Table~\ref{tab:comparison} reinforces these results numerically, showing a
clear fit by the best parameters across the board.

\begin{table}
    \centering
    \resizebox{\textwidth}{!}{\begin{tabular}{lrrr}
\toprule
{} &    Observed &  Best simulated &  Worst simulated \\
\midrule
mean &    7.703670 &        6.751885 &         3.845943 \\
std  &   11.865440 &        7.134836 &         3.869550 \\
min  &   -0.020833 &        0.000025 &         0.000013 \\
25\%  &    1.490278 &        1.807334 &         1.103671 \\
50\%  &    4.195139 &        4.452596 &         2.655732 \\
75\%  &    8.930556 &        9.252244 &         5.330906 \\
max  &  224.927778 &       90.822876 &        47.801514 \\
\bottomrule
\end{tabular}
}
    \caption{A comparison of the observed data, and the best and worst simulated
        data based on the model parameters and summary statistics for length of
    stay (LOS).}\label{tab:comparison}
\end{table}
