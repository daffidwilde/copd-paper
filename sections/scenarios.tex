\section{Adjusting the queuing model}\label{sec:scenarios}

With the queuing model established and validated in Section~\ref{sec:model}, an
investigation into the capabilities of the system can be conducted by tweaking
the parameters of the model. This section is comprised of several `what-if'
scenarios --- a classic component of healthcare operational research --- under
this novel parameterisation. The outcomes of interest in this work are server
(resource) utilisation and system times as these capture the driving forces of
cost and flow as well as the overall state of the system, its staff and its
patients.  Specifically, the objective of these experiments is to address the
following questions:
\begin{itemize}
    \item How would the system be affected by a change in overall patient
        arrivals?
    \item How is the system affected by a change in resource availability (i.e.\
        a change in \(c\))?
    \item How is the system affected by patients moving between clusters?
\end{itemize}

Owing to the nature of the observed data, the queuing model parameterisation
and its assumptions, the effects of each scenario are given in relative terms
with respect to the base case. The base case being those results generated from
the best parameter set recorded in Table~\ref{tab:comparison}. In particular,
each piece of data in each scenario is simply scaled by the corresponding median
value in the base case meaning that a value of 1 is `normal'.

As mentioned in Section~\ref{sec:intro}, the source code used throughout this
work is available online and has been archived. %TODO Add citation for repo
In addition to this, the datasets generated from the simulations in this section
have been archived. %TODO archive and add citation


\subsection{Changes to overall patient arrivals}\label{subsec:arrivals}

Changes in overall patient arrivals to a queue reflect real-world scenarios
where some stimulus is improving (or worsening) the condition of the patient
population. Examples of positive stimuli include increased community care and
campaigns against harmful behaviours such as smoking. Within this model, overall
patient arrivals are altered using a scaling factor denoted by
\(\sigma\in\mathbb{R}\). This scaling factor is applied to the model by
multiplying each cluster's arrival rate by \(\sigma\). That is, for cluster
\(i\), its new arrival rate, \(\hat\lambda_i\), is given by:
\begin{equation}\label{eq:lambda}
    \hat\lambda_{i} = \sigma\lambda_i
\end{equation}

\begin{figure}
    \centering
    \begin{subfigure}{.5\imgwidth}
        \includegraphics[width=\linewidth]{lambda_time}
        \caption{}\label{fig:lambda_time}
    \end{subfigure}\hfill%
    \begin{subfigure}{.5\imgwidth}
        \includegraphics[width=\linewidth]{lambda_util}
        \caption{}\label{fig:lambda_util}
    \end{subfigure}
    \caption{%
        Plots of \(\sigma\) against relative (\subref{fig:lambda_time})~system
        time and (\subref{fig:lambda_util})~server utilisation.
    }\label{fig:lambda}
\end{figure}

Figure~\ref{fig:lambda} shows the effects of changing patient arrivals on
(\subref{fig:lambda_time})~relative system times and
(\subref{fig:lambda_util})~relative server utilisation over values of \(\sigma\)
from \input{tex/lambda_scaling_min} to \input{tex/lambda_scaling_max} at a
precision of \input{tex/lambda_scaling_step}. Specifically, each plot in the
figure (and the subsequent figures in this section) shows the median and
interquartile range (IQR) of each relative attribute. These metrics provide an
insight into the experience of the average user (or server) in the system, and
in the stability or variation of the body of users (servers).

The overarching implication of these plots is that the system has some slack.
Looking at Figure~\ref{fig:lambda_time}, for instance, the relative system times
(i.e.\ the relative length of stay for patients) remains unchanged up to
\(\sigma \approx 1.2\), or an approximate 20\% increase in arrivals of COPD
patients. Beyond that, relative system times rise to an untenable point where
the median time becomes orders of magnitude above the norm.

However, Figure~\ref{fig:lambda_util} shows that the situation for the system's
resources reaches its worst case near the start of that spike in relative system
times (at \(\sigma \approx 1.4\)). That is, the median server utilisation
reaches a maximum (this corresponds to constant utilisation) at this point and
the variation in server utilisation disappears entirely.


\subsection{Changes to resource availability}\label{subsec:resources}

As is discussed in Section~\ref{sec:model}, the resource availability of the
system is captured by the number of parallel servers in the system, \(c\).
Therefore, to modify the overall resource availability, only the number of
servers need be changed. This kind of sensitivity analysis is usually done to
determine the opportunity cost of adding service capacity to a system, e.g.\
would adding \(n\) servers sufficiently increase efficiency without exceeding
a budget?

To reiterate the beginning of this section, all suitable parameters are given in
relative terms. This includes the number of servers here. By doing this, the
changes in resource availability are more easily seen, and do away with any
concerns as to what a particular number of servers exactly reflects in the real
world.

\begin{figure}
    \centering
    \begin{subfigure}{.5\imgwidth}
        \includegraphics[width=\linewidth]{servers_time}
        \caption{}\label{fig:servers_time}
    \end{subfigure}\hfill%
    \begin{subfigure}{.5\imgwidth}
        \includegraphics[width=\linewidth]{servers_util}
        \caption{}\label{fig:servers_util}
    \end{subfigure}
    \caption{%
        Plots of the relative number of servers against relative
        (\subref{fig:servers_time})~system time and
        (\subref{fig:servers_util})~server utilisation.
    }\label{fig:servers}
\end{figure}

Figure~\ref{fig:servers} shows how the relative resource availability affects
relative system times and server utilisation. In this scenario, the relative
number of servers took values from \input{tex/num_servers_change_min} to
\input{tex/num_servers_change_max} at steps of
\(2.9 \times 10^{-2}\)% --- this is equivalent to a step size of 1
in the actual number of servers. Overall, these figures fortify the
claim from the previous scenario that there is some room to manoeuvre in the
system but pressing on those boundaries results in massive changes to both
resource requirements and system times.

In Figure~\ref{fig:servers_time} this amounts to a maximum of 20\% slack in
resources before relative system times are affected; further reductions quickly
result in a potentially tenfold increase in the median system time, and up to 50
times once resource availability falls by 50\%. Moreover, the variation in the
body of the relative times (i.e.\ the IQR) decreases as resource availability
decreases. The reality of behaviour like that is that patients arriving at a
hospital are forced to consume larger amounts of resources regardless of their
condition, putting a strain on the system.

Meanwhile, it appears that there is no tangible change in relative system times
given an increase in the number of servers. This indicates that the model
carries sufficient resources to cater to the population under normal
circumstances, and that adding service capacity will not improve system times.

Again, Figure~\ref{fig:servers_util} shows that there is a substantial change in
the variation in the utilisation of the servers. In this case, the variation
dissipates as resource levels fall and increases as they increase. While the
relationship between real hospital resources and the number of servers is not
exact, having variation in server utilisation would suggest that parts of the
system may be configured or partitioned away in the case of some significant
public health event (such as a global pandemic) without overloading the system.


\subsection{Moving patients between clusters}\label{subsec:moving}

In order to model the effects of patients moving between two clusters, the
assumption is that services remain the same (and so does each cluster's \(p_i\))
but their arrival rates are altered according to some transfer proportion.
Consider two clusters indexed at \(i, j\), and their respective arrival rates,
\(\lambda_i, \lambda_j\), and let \(\delta \in [0, 1]\) denote the proportion of
arrivals to be moved from cluster \(i\) to cluster \(j\). Then the new arrival
rates for each cluster, denoted by \(\hat\lambda_i, \hat\lambda_j\)
respectively, are:
\begin{equation}\label{eq:moving}
    \hat\lambda_i = \left(1 - \delta\right) \lambda_i
    \quad \text{and} \quad
    \hat\lambda_j = \delta\lambda_i + \lambda_j
\end{equation}

By moving patient arrivals between clusters in this way, the overall arrivals
are left the same since the sum of the arrival rates is the same. Hence, the
(relative) effect on server utilisation and system time can be measured
independently.

\begin{figure}
    \centering
    \includegraphics[width=\imgwidth]{moving_time}
    \caption{%
        Plots of proportions of each cluster moving to another against relative
        system time.
    }\label{fig:moving_time}
\end{figure}

\begin{figure}
    \centering
    \includegraphics[width=\imgwidth]{moving_util}
    \caption{%
        Plots of proportions of each cluster moving to another on relative
        server utilisation.
    }\label{fig:moving_util}
\end{figure}
