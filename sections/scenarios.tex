\section{Adjusting the queuing model}\label{sec:scenarios}

With the queuing model established and validated in
Section~\ref{sec:wasserstein}, an investigation into the parameters of the model
can be conducted. This section is comprised of several `what-if' scenarios --- a
classic component of healthcare operational research --- under this novel
parameterisation. The outcomes of interest in this work are server (resource)
utilisation and system times as these capture the both the driving forces of
cost and the overall state of the system. Specifically, the objective of these
experiments is to address the following questions:
\begin{itemize}
    \item How would server utilisation be affected by a change in overall
        patient arrivals?
    \item What are the sensitivities of mean system times and server utilisation
        based on a change in \(c\)?
    \item How is the system affected by patients moving between clusters?
\end{itemize}

\subsection{Changes to overall patient arrivals}\label{subsec:arrivals}

\subsection{Changes to resource availability}\label{subsec:resources}

\subsection{Moving patients between clusters}\label{subsec:moving}

