\section{Conclusion}\label{sec:conclusion}

This work presents a novel approach to investigating a healthcare population
that encompasses the topics of segmentation analysis, queuing models, and the
recovery of queuing parameters from incomplete data. This is done despite common
limitations in operational research with regard to the availability of
fine-grained data, and this work only uses administrative hospital spell data
from patients presenting COPD from the Cwm Taf Morgannwg UHB.\

By considering a variety of attributes present in the data, and engineering
some, an effective clustering of the spell population is identified that
successfully feeds into a multi-class, \(M/M/c\) queue to model a hypothetical
COPD ward. With this model, a number of insights are gained by investigating
purposeful changes in the parameters of the model that have the potential to
inform actual public health policy.

In particular, since neither the resource capacity of the system or the clinical
processes of the spells are evident in the data, service times and resource
levels are not available. However, length of stay is. Using what is available,
this work assumes that mean service times can be parameterised using mean
lengths of stay. By using the Wasserstein distance to compare the distribution
of the simulated lengths of stay data with the observed data, a best performing
parameter set is found via a parameter sweep.

This parameterisation ultimately recovers a surrogate for service times for each
cluster, and a common number of servers to emulate resource availability. The
parameterisation itself offers its strengths by being simple and effective.
Despite its simplicity, a good fit to the observed data is found, and --- as is
evident from the closing section of this work --- substantial and useful
insights can be gained into the needs of the population being studied.

This analysis, and the formation of the entire model, in effect, considers all
types of patient arrivals and how they each impact the system in terms of
resource capacity and length of stay. By investigating scenarios into changes in
both overall patient arrivals and resource capacity, it is clear that there is
no quick solution to be employed from within the hospital to improve COPD
patient spells.  The only effective, non-trivial intervention is to improve the
overall health of the patients arriving at the hospital. This is shown by moving
patient arrivals between clusters. In reality, this would correspond to an
external, preventative policy that improves the overall health of COPD patients.
