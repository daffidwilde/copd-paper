\section{Introduction}\label{sec:intro}

Population health research is becoming increasingly based on data-driven methods
(as opposed to those designed by clinical experts) through the advent of
accessible software and an abundance of electronic data. To better understand
the needs and behaviours of a population, it can be helpful to find an
appropriate segmentation of that population; such a segmentation allows for
finer-grained analysis of groups in the population that share some form of
homogeneity. One commonly used analysis is of patient flow and queuing systems.

However, this process relies heavily on detailed data --- about both the system
and the population within that system --- which may limit research where
sophisticated data pipelines are not yet in place. This work demonstrates how
this issue may be overcome using administrative, spell-level hospital data to
build a meso-level population clustering and a multi-class queuing model. 

% TODO Image of process: administrative data -> extract service parameters ->
%                        validate parameters with Wasserstein distance ->
%                        continue with queuing model as normal

% TODO Brief literature review of:
%       - clustering patient corpus
%       - reverse-engineering model parameters in healthcare
%       - queuing in healthcare

% TODO Condensed analysis of clustering
%       - choosing k and a clustering algorithm
%       - length of stay
%       - clinical indicators
%       - comparison with treatment segmentation
