\begin{abstract}
    This work uses a data-driven approach to analyse how the resource
    requirements of patients with chronic obstructive pulmonary disease (COPD)
    may change, and quantifies how those changes affect the strains of the
    hospital system the patients interact with. This is done using a novel
    combination of often distinct modes of analysis: segmentation, operational
    queuing theory, and the recovery of parameters from incomplete data. By
    combining these methods as presented here, it is demonstrated how the
    potential limitations around the availability of fine-grained data can be
    overcome in operational research. That is, despite using only administrative
    data, useful operational results are found.

    The paper begins by finding an effective clustering of the population from
    this granular data that feeds into a multi-class \(M/M/c\) model, whose
    parameters are recovered from the data via parameterisation and the
    Wasserstein distance. This model is then used to conduct an informative
    analysis of the underlying queuing system and the needs of the population
    under study through several `what-if' scenarios.

    The particular analyses used to form and study this model considers, in
    effect, all types of patient arrivals and how those types impact the system.
    With that, this study finds that there are no quick solutions to reduce the
    impact of COPD patients on the system, including adding capacity to the
    system. In fact, the only effective intervention to reduce the strains of
    the COPD population is to enact some external policy that directly improves
    the overall health of the COPD population before arriving at the hospital.
\end{abstract}
