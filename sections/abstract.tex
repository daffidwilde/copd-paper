\begin{abstract}
    This work analyses how the resource requirements of a population of COPD
    patients in the South Wales area may change, and quantifies how those
    changes effect the strains on a hospital system. This is done using a novel
    combination of often distinct modes of analysis: segmentation, operational
    queuing theory, and the recovery of queuing parameters from incomplete data.
    This is done despite common limitations in operational research with regard
    to the availability of fine-grained data as this work only uses
    administrative hospital spell data from those patients.

    An effective clustering of the population is found from this granular data
    and feeds into a multi-class \(M/M/c\) model, the parameters of which are
    recovered from the available data via a simple parameterisation and the
    Wasserstein distance. This model is then used to conduct a substantial and
    informative analysis of the underlying queuing system and the needs of the
    population under study through the simulation of several `what-if'
    scenarios.

    The particular set of analyses used to form and study this model
    considers, in effect, all types of patient arrivals as well how those types
    each impact the system. With that, this study finds that there are no quick
    solutions to reduce the impact of COPD patients on the system, including
    adding capacity to the system. In fact, the only effective intervention to
    reduce the strains of the COPD population is to enact some external policy
    that directly improves the overall health of the COPD population before
    arriving at the hospital.
\end{abstract}
