\begin{abstract}
    Data-driven research in healthcare settings is paramount. This work
    investigates a population of COPD patients in the South Wales area and
    encompasses the topics of segmentation analysis, queuing modelling, and the
    recovery of queuing parameters from incomplete data. This is done despite
    common limitations in operational research with regard to the availability
    of fine-grained data as this work only uses administrative hospital spell
    data from those patients.

    An effective clustering of the population is found from this granular data
    and feeds into a multi-class \(M/M/c\) model, the parameters of which are
    recovered from the available data via a simple parameterisation and the
    Wasserstein distance. This model is then used to conduct a substantial and
    informative analysis of the underlying queuing system and the needs of the
    population under study through the simulation of several `what-if'
    scenarios.
\end{abstract}
