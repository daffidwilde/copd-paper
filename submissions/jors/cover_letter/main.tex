\documentclass[11pt]{letter}

\setlength{\oddsidemargin}{-.10in}
\setlength{\textwidth}{6.7in}
\setlength{\textheight}{11in}
\setlength{\topmargin}{-.40in}

\begin{document}

\signature{The authors}

\begin{letter}{}

\textbf{
  Segmentation analysis and the recovery of queuing parameters via the
  Wasserstein distance: a study of administrative data for patients with chronic
  obstructive pulmonary disease
}

To whom it may concern,

We present a novel approach for dealing with lack of data as inputs to queuing
theoretic models. All research software designed for the work
and all data is made available for all to use according to the best open
scientific principles.

The paper also presents a novel application of queue modelling within healthcare
by considering COPD patients. Clustering of patients using an unsupervised
algorithm ensures accurate groupings informed by the data. This is used in
combination with the queuing model to conclude that no quick wins are possible
for the particular healthcare system and in fact a public health
intervention is required to improve the general health within the population.

A precise aspect of the work here is that the clustering algorithm identifies a
specific group of patients for which an intervention would be most
beneficial to the efficiency of the healthcare system.

All of the above is made possible by the major contribution of the paper:
dealing with lack of data. Indeed, as is often the case in healthcare models
actual service length data is not available due to delay of admission or
delay of discharge. The approach here is to use the Wasserstein distance as a
metric on length of stay distributions to recover the service length from
the data.

This approach could be readily used in similar modelling exercises
where accurate service length data is not available.

\closing{Sincerely,}

\end{letter}

\end{document}
